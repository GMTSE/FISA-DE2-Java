\documentclass[English,t,% 't' (resp. 'c') places text vertically at top/center of the slides
% PDF settings
hyperref={%
    pdftitle={FISA-DE2 OOP in Java},%
    pdfauthor={Guillaume Muller},%
    pdfsubject={Writing Clean Java Code},%
    pdfkeywords={OOP,Java,Clean Code}%
    },%
% To load many pre-defined color names
xcolor={pdftex,svgnames} % dvipsnames, dvipsnames*, svgnames, svgnames*, x11names,
]{beamer}
\usetheme{Copenhagen} % AnnArbor,Antibes,Bergen,Berkeley,Berlin,Boadilla,CambridgeUS,Copenhagen,Darmstadt,Dresden,Frankfurt,Goettingen,Hannover,Ilmenau,JuanLesPins,Luebeck,Madrid,Malmoe,Marburg,Montpellier,PaloAlto,Pittsburgh,Rochester,Singapore,Szeged,Warsaw,boxes,default

% Correct French/English indentation and splitting of words
\usepackage{babel}

% Correct management of accentuated chars in input file
\usepackage[utf8]{inputenc}
%\usepackage[utf8]{inputenc}

% Correct font for the generation of docs with accentuated chars
\usepackage[T1]{fontenc}      % Can handle hyphenation of words with accented characters
%%\usepackage[OT1]{fontenc}   % Might generated bad looking PDFs

% Access to many maths symbols
\usepackage{amsthm}
\usepackage{amsmath}
\usepackage{amsfonts}

% Insertion of images generated by external tools
\usepackage{graphicx}

% To generate pretty & scalable images directly in LaTeX
\usepackage{tikz}
% \draw[decorate,decoration={coil,amplitude=1.5cm, segment length=.4cm}] (5,5.5) -- (5,1.5) ;
\usetikzlibrary{tikzmark}
\usetikzlibrary{calc}
\usetikzlibrary{decorations.text}
\usetikzlibrary{decorations.shapes}
\usetikzlibrary{decorations.pathmorphing,snakes}

\def\me{Guillaume \textsc{Muller}}

% To print numbers correctly
\usepackage{numprint}

% To position text blocks absolutely
\usepackage[absolute,overlay]{textpos}

% For UML diagrams
\usepackage{tikz-uml}

% To be able to strikeout text
\usepackage[normalem]{ulem}

% To be able to insert code listing
\usepackage{listings}

\definecolor{dkgreen}{rgb}{0,0.6,0}
\definecolor{gray}{rgb}{0.5,0.5,0.5}
\definecolor{mauve}{rgb}{0.58,0,0.82}

\lstset{frame=none,
  language=Java,
  aboveskip=1mm,
  belowskip=1mm,
  showstringspaces=false,
  columns=flexible,
  basicstyle={\tiny \ttfamily},
  numbers=left,
  numberstyle=\tiny\color{gray},
  keywordstyle=\color{blue},
  commentstyle=\color{dkgreen},
  stringstyle=\color{mauve},
  breaklines=true,
  breakatwhitespace=true,
  tabsize=2
}

% Info for title page
\title[Clean Code]{Writing clean Java code}
\logo{\includegraphics[width=1cm]{images00/logo_tse.png}}
\author[\me{}]{\me{}}
\institute[TSÉ + LHC]{
  \inst{Telecom Saint-Étienne, Laboratoire Hubert-Curien}%
}
\date[10:07/2019]{7~october~2019}
\subject{OOP in Java}


%%%%%%%%%%%%%%%%%%%%%%%%%%%%%%%%%%%%%%%%%%%%%%%%%%%%%%%%%%%%%%%%%%%%%%
\begin{document}

% The title page
\begin{frame}
  \titlepage
\end{frame}

%%%%%%%%%%%%%%%%%%%%%%%%%%%%%%%%%%%%%%%%%%%%%%%%%%%%%%%%%%%%%%%%%%%%%%
\begin{frame}{Writing re-usable code}

  \begin{itemize}
    \item Java By Comparison
    { \tiny \url{https://www.amazon.com/Java-Comparison-Become-Craftsman-Examples/dp/1680502875/} } \\
    { \tiny \url{https://github.com/GMTSE/ProjetsJavaMaterial/blob/master/JavaByComparisonSumUp.md} }
{\small
      \begin{itemize}
        \item Rule$\sharp$5: always check for \textbf{null args}
        \item Rule$\sharp$8/$\sharp$14: \textbf{group} code / \textbf{indent}
        \item Rule$\sharp$72: log in file not console
        \item Rule$\sharp$11: favor for-each
        \item Rule$\sharp$15: use string format
        \item Rule$\sharp$68: 1 \textbf{code style} for the team
        \item Rule$\sharp$21: template for \textbf{comments}
        \item Rule$\sharp$29: select \textbf{good names}
        \item Rule$\sharp$34: catch most specific Exception
        \item Rule$\sharp$39: close resources
        \item Rule$\sharp$42: template for Unit Tests
        \item Rule$\sharp$50: treat edge test cases
        \item Rule$\sharp$54: favor immutable
        \item Rule$\sharp$64: use optionals rather than nulls
      \end{itemize}
}
  \end{itemize}

\end{frame}


\end{document}
